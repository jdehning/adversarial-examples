% ****** Start of file apssamp.tex ******
%
%   This file is part of the APS files in the REVTeX 4.1 distribution.
%   Version 4.1r of REVTeX, August 2010
%
%   Copyright (c) 2009, 2010 The American Physical Society.
%
%   See the REVTeX 4 README file for restrictions and more information.
%
% TeX'ing this file requires that you have AMS-LaTeX 2.0 installed
% as well as the rest of the prerequisites for REVTeX 4.1
%
% See the REVTeX 4 README file
% It also requires running BibTeX. The commands are as follows:
%
%  1)  latex apssamp.tex
%  2)  bibtex apssamp
%  3)  latex apssamp.tex
%  4)  latex apssamp.tex
%
\documentclass[%
 reprint,
%superscriptaddress,
%groupedaddress,
%unsortedaddress,
%runinaddress,
%frontmatterverbose, 
%preprint,
%showpacs,preprintnumbers,
%nofootinbib,
%nobibnotes,
%bibnotes,
 amsmath,amssymb,
 aps,
%pra,
%prb,
%rmp,
%prstab,
%prstper,
%floatfix,
]{revtex4-1}

\usepackage[T1]{fontenc}
\usepackage[utf8]{inputenc}
\usepackage{graphicx}% Include figure files
\usepackage{dcolumn}% Align table columns on decimal point
\usepackage{bm}% bold math
%\usepackage{hyperref}% add hypertext capabilities
%\usepackage[mathlines]{lineno}% Enable numbering of text and display math
%\linenumbers\relax % Commence numbering lines

%\usepackage[showframe,%Uncomment any one of the following lines to test 
%%scale=0.7, marginratio={1:1, 2:3}, ignoreall,% default settings
%%text={7in,10in},centering,
%%margin=1.5in,
%%total={6.5in,8.75in}, top=1.2in, left=0.9in, includefoot,
%%height=10in,a5paper,hmargin={3cm,0.8in},
%]{geometry}

\begin{document}

\preprint{APS/123-QED}

\title{Adverserial Examples}% Force line breaks with \\
%\thanks{A footnote to the article title}%

\author{Jan Plank}
\email{janhendrik.plank@stud.uni-goettingen.de}
% \altaffiliation[Also at ]{Physics Department, XYZ University.}%Lines break automatically or can be forced with \\
\author{Philipp Höhne}%
 \email{philipp.hoehne@stud.uni-goettingen.de}
%

%\collaboration{MUSO Collaboration}%\noaffiliation

\author{Jonas Dehning}
\email{j.dehning@stud.uni-goettingen.de}
\affiliation{
Universität Göttingen
}%

\date{\today}% It is always \today, today,
             %  but any date may be explicitly specified

\begin{abstract}
In this work we generate adversarial examples for convolutional networks trained on two different datasets. Two different methods are compared for the generation: the first is to add a small ``noise'' in the direction of the gradient of the loss function, the other is to minimize a custom function which allows to find an example nearer to the original image than with the gradient method. Than the robustness of the network to adversarial examples are compared as function of the depth of networks, for the two datasets.


%\begin{description}
%\item[Usage]
%Secondary publications and information retrieval purposes.
%\item[PACS numbers]
%May be entered using the \verb+\pacs{#1}+ command.
%\item[Structure]
%You may use the \texttt{description} environment to structure your abstract;
%use the optional argument of the \verb+\item+ command to give the category of each item. 
%\end{description}
\end{abstract}

%\pacs{Valid PACS appear here}% PACS, the Physics and Astronomy
                             % Classification Scheme.
%\keywords{Suggested keywords}%Use showkeys class option if keyword
                              %display desired
\maketitle

%\tableofcontents

\section{Introduction}

In general machine learning techniques are vulnerable to adversarial examples, meaning that they misclassify examples which are only slightly different to correctly classified ones. In many cases different machine learning techniques, even if trained on different subsets of the training data, misclassify the same adversarial examples. Thus adversarial examples seem to reveal a fundamental weakness of those machine learning techniques. Furthermore this suggests that these techniques are unable to learn the true underlying concepts for the correct classification, although they perform extremely well on the training data. They work well for naturally occurring data but fail if they visit points in space that are unlikely to occur in the data. This is problematic for computer vision where it is assumed that the Euclidean distance approximates the perceptual distance.

In course of the increasing use of neural networks the understanding of their weaknesses is of growing interest. To be able to better understand the effects of adversarial examples on neural networks one first has to find ways how to create these. That's why we test two

\subsection{Convolutional Neural Networks}

Convolutional neural networks are used to categorize images. The are composed of successive convolutional and pooling layers. In the convolutional layer each pixel (neuron) gets the input of the surrounding pixels of the previous layer or the image, and multiply it by a matrix, thus making a convolution. The convolutional matrix is the same for the whole image but typically for each convolutional layer a set of matrices each one applied to the whole image, also called filters, are used. 

\subsubsection{Wide text (A level-3 head)}
The \texttt{widetext} environment will make the text the width of the
full page, as on page~\pageref{eq:wideeq}. (Note the use the
\verb+\pageref{#1}+ command to refer to the page number.) 
\paragraph{Note (Fourth-level head is run in)}
The width-changing commands only take effect in two-column formatting. 
There is no effect if text is in a single column.

\subsection{\label{sec:citeref}Citations and References}
A citation in text uses the command \verb+\cite{#1}+ or
\verb+\onlinecite{#1}+ and refers to an entry in the bibliography. 
An entry in the bibliography is a reference to another document.

\subsubsection{Citations}
Because REV\TeX\ uses the \verb+natbib+ package of Patrick Daly, 
the entire repertoire of commands in that package are available for your document;
see the \verb+natbib+ documentation for further details. Please note that
REV\TeX\ requires version 8.31a or later of \verb+natbib+.

\paragraph{Syntax}
The argument of \verb+\cite+ may be a single \emph{key}, 
or may consist of a comma-separated list of keys.
The citation \emph{key} may contain 
letters, numbers, the dash (-) character, or the period (.) character. 
New with natbib 8.3 is an extension to the syntax that allows for 
a star (*) form and two optional arguments on the citation key itself.
The syntax of the \verb+\cite+ command is thus (informally stated)
\begin{quotation}\flushleft\leftskip1em
\verb+\cite+ \verb+{+ \emph{key} \verb+}+, or\\
\verb+\cite+ \verb+{+ \emph{optarg+key} \verb+}+, or\\
\verb+\cite+ \verb+{+ \emph{optarg+key} \verb+,+ \emph{optarg+key}\ldots \verb+}+,
\end{quotation}\noindent
where \emph{optarg+key} signifies 
\begin{quotation}\flushleft\leftskip1em
\emph{key}, or\\
\texttt{*}\emph{key}, or\\
\texttt{[}\emph{pre}\texttt{]}\emph{key}, or\\
\texttt{[}\emph{pre}\texttt{]}\texttt{[}\emph{post}\texttt{]}\emph{key}, or even\\
\texttt{*}\texttt{[}\emph{pre}\texttt{]}\texttt{[}\emph{post}\texttt{]}\emph{key}.
\end{quotation}\noindent
where \emph{pre} and \emph{post} is whatever text you wish to place 
at the beginning and end, respectively, of the bibliographic reference
(see Ref.~[\onlinecite{witten2001}] and the two under Ref.~[\onlinecite{feyn54}]).
(Keep in mind that no automatic space or punctuation is applied.)
It is highly recommended that you put the entire \emph{pre} or \emph{post} portion 
within its own set of braces, for example: 
\verb+\cite+ \verb+{+ \texttt{[} \verb+{+\emph{text}\verb+}+\texttt{]}\emph{key}\verb+}+.
The extra set of braces will keep \LaTeX\ out of trouble if your \emph{text} contains the comma (,) character.

The star (*) modifier to the \emph{key} signifies that the reference is to be 
merged with the previous reference into a single bibliographic entry, 
a common idiom in APS and AIP articles (see below, Ref.~[\onlinecite{epr}]). 
When references are merged in this way, they are separated by a semicolon instead of 
the period (full stop) that would otherwise appear.

\paragraph{Eliding repeated information}
When a reference is merged, some of its fields may be elided: for example, 
when the author matches that of the previous reference, it is omitted. 
If both author and journal match, both are omitted.
If the journal matches, but the author does not, the journal is replaced by \emph{ibid.},
as exemplified by Ref.~[\onlinecite{epr}]. 
These rules embody common editorial practice in APS and AIP journals and will only
be in effect if the markup features of the APS and AIP Bib\TeX\ styles is employed.

\paragraph{The options of the cite command itself}
Please note that optional arguments to the \emph{key} change the reference in the bibliography, 
not the citation in the body of the document. 
For the latter, use the optional arguments of the \verb+\cite+ command itself:
\verb+\cite+ \texttt{*}\allowbreak
\texttt{[}\emph{pre-cite}\texttt{]}\allowbreak
\texttt{[}\emph{post-cite}\texttt{]}\allowbreak
\verb+{+\emph{key-list}\verb+}+.

\end{document}
%
% ****** End of file apssamp.tex ******